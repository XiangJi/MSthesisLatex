\chapter*{Abstract} 
%%%%%For now just place holder
In a meteorology data-distribution application, streams of files are served to hundreds of receivers every day on unicast TCP connections. Software Defined Networking (SDN) offers a more-scalable solution in which 
a rate-guaranteed Layer-2 (L2) multipoint virtual topology can be provisioned to have switches perform Ethernet-frame
multicasting to support this application. A characterization
of the file streams shows that file sizes and
file inter-arrival times are both right skewed.
The objective of this thesis is to design an algorithm for determining
the rate of the Layer-2 multipoint virtual topology, and the size
of the sending-host buffer, based on
traffic characteristics of the file streams and performance requirements.
Furthermore, the traffic characteristics are not exactly the same
from day-to-day. An empirical method is proposed to determine the ideal rate and buffer size based on a day's traffic, which are then used along with the current rate and buffer
size in an Exponential Weighted Moving Average (EWMA) scheme to determine the rate and buffer size for the next day. Our method was
evaluated using metadata obtained for the top five file-streams of this meteorological data distribution, and found to be effective.

This thesis also describes our experience with deploying a multi-domain SDN that supports dynamic L2 path service, and offers insights gained from this experience. OpenFlow switches with two controllers, Open Exchange Software Suite (OESS) to perform intra-domain topology discovery and path provisioning actions, and On-Demand Secure Circuits and Advance Reservation System (OSCARS) for inter-domain provisioning, were deployed in several university campuses, and regional and core research-and-education networks (RENs). Our experience demonstrated that this architecture can support global-scale multi-domain dynamic L2 path service.  We identified modifications required to the protocols and controllers for improving user experience and scalability of this dynamic L2 path service. We also developed a methodology for provisioning inter-domain multipoint VLANs, and demonstrated the successful use of these VLANs for a multicast application.


